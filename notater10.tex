\documentclass[12p, a4paper]{article}
\title{Chapter 10: Managing memory and low-level data structures}
\begin{document}
\maketitle
We will in this chapter be introduced to \textbf{\textit{low-level}} programming that resembles how the language itself i built. Using low-level techniques is one skill that is very useful in C++.\\
\\

An \textbf{\textit{array}} is a kind of container, similar to a vector but less powerful. A \textbf{\textit{pointer}} is a kind of random-sccess iterator that is essential for accessing elements of arrays, and has other uses as well. \\
\\

A pointer can be said to be the \textbf{\textit{address}} of an object. Every distinct object has a unique address, which denotes the part of the computer's memory that contains the object. If \texttt{x} is an object, then \texttt{\&x} is the address of that object, and if \texttt{p} is the address of an object, then \texttt{*p} is the object itself. The \texttt{\&} is an \textbf{\textit{address operator}} and the \texttt{*} is a \textbf{\textit{dereference operator}}.

\end{document}