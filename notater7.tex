\documentclass[12p, a4paper]{article}
\begin{document}
Instead of storing data in a sequential container, we can use an \textbf{\textit{associative container}}. Such containers automatically arrange their elements into a sequence that depends on the values of the alements themselves, rather than the sequence in which we inserted them. They exploit this ordering to let us locate particular elements much more quickly than the sequential containers, without having to keep the container ordered by ourselves. \\

Some parameters can have a \textbf{\textit{default argument}}. When this is given to a parameter, the user can choose to give it a new argument or use the default one. When we for example say xref ... = split, the user can use the program in two ways. We can either say\\
\\
xeref(cin); // \textit{Uses} split \textit{to find words in the input stream}  \\
xref(cin, find\_urls); // \textit{uses the function named} find\_urls \textit{to find words}
\end{document}