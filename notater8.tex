\documentclass[12p, a4paper]{article}
\begin{document}
If we don't know what a functions' argument or result types will be until we use it, we call it a \textbf{\textit{generic function}}. The ability to use and create generic functions is a key feature of the C++ language. The language feature that implements generic functions is called \textbf{\textit{template functions}}. They let us write a single definition for a family of functions - or types - that behave similarly, except for differences that we can attribute the types of their \textbf{\textit{template parameters}}.\\
\\

\textbf{\textit{Type parameters}} operate much like function parameters. They define names that can be used within the scope of the function. However, type parameters refer to types, not to variables. For \texttt{template <class T>}, when T appears in the function, the implementation will assume that T is a type. \\
\\

The library defines five \textbf{\textit{iterator categories}}, each one of which corresponds to a specific collection of iterator operations. These categories classify the kind of iterator that each of the library containers provide.\\
\\


\end{document}